\documentclass[10pt,a4paper]{article}
\usepackage[latin1]{inputenc}
\usepackage{amsmath}
\usepackage{amsfonts}
\usepackage{amssymb}
\usepackage{graphicx}
\usepackage{accsupp}

\title{Practical work: Command Line interface \\
}

\begin{document}
	
	\maketitle
	
	\section{Introduction}
	% Open a terminal 
	The content of this practical work is presented by increasing order of difficulty. It is not mandatory to finish the subject. However, each student is expected to go up until the end of step 10.
	

	
	\section{Exercises with the command line interface}
	
	\subsection{Initialization}
	To begin the exercise, perform the following tasks:
 	\begin{enumerate}
		\item Using a Web Browser, download the archive \texttt{command\_line\_archive.zip} 
		\item Using your file manager, go to the directory were the file  \texttt{command\_line\_archive.zip} is stored
		\item Extract the archive (right click extract)
		\item Open a terminal in this directory.
		
	\end{enumerate}

\begin{center}
\bf	From this point all the manipulation have to be done using the terminal! \\
 (You are allowed to use mouse and keyboard to browse the lecture slides and other documentation).
\end{center}

	In the terminal, execute the program 'program\_lvl1.sh' by typing the following command and pressing enter.
	\BeginAccSupp{ActualText=I don't want you to copy-paste those commands.}
	\begin{verbatim}
		./program_lvl1.sh
	\end{verbatim}
	\EndAccSupp{}
	
	\subsection{Next steps}
	
	Follow the instructions and solve as many exercises as possible !
	
	
\end{document}